\documentclass[a4paper,12pt]{article}

% Paquetes necesarios
\usepackage[utf8]{inputenc}     % Codificación de caracteres
\usepackage[spanish]{babel}     % Idioma del documento
\usepackage{amsmath, amssymb}   % Paquetes para símbolos y ecuaciones matemáticas
\usepackage{graphicx}           % Para insertar imágenes
\usepackage{caption}            % Para leyendas
\usepackage{subcaption}         % Para subfiguras
\usepackage{geometry}           % Control de márgenes
\usepackage{hyperref}           % Enlaces dentro del documento
\geometry{top=2cm, bottom=2cm, left=2cm, right=2cm}

% Datos del documento
\title{Trabajo Práctico 1}
\author{Alejo Ordoñez - 108397}
\date{Jueves 26 de Septiembre, 2024}

\begin{document}

\maketitle

\newpage
\tableofcontents


\newpage
\section{Introducción}

En este informe se presentan varios resultados obtenidos empíricamente a partir del entrenamiento de una red de Hopfield.\textit{}


\begin{figure}[h]  % Inicia el entorno para las imágenes
    \centering
    % Primera fila de imágenes
    \begin{subfigure}{0.3\textwidth}
        \centering
        \includegraphics[width=\linewidth]{imágenes/paloma.png}  % Reemplaza por el nombre de tus imágenes
        % \caption{Imagen 1}
        % \label{fig:imagen1}
    \end{subfigure}
    \hfill
    \begin{subfigure}{0.3\textwidth}
        \centering
        \includegraphics[width=\linewidth]{imágenes/panda.png}
    \end{subfigure}
    \hfill
    \begin{subfigure}{0.3\textwidth}
        \centering
        \includegraphics[width=\linewidth]{imágenes/perro.png}
    \end{subfigure}
    
    % Segunda fila de imágenes
    \begin{subfigure}{0.3\textwidth}
        \centering
        \includegraphics[width=\linewidth]{imágenes/quijote.png}
    \end{subfigure}
    \hfill
    \begin{subfigure}{0.3\textwidth}
        \centering
        \includegraphics[width=\linewidth]{imágenes/torero.png}
    \end{subfigure}
    \hfill
    \begin{subfigure}{0.3\textwidth}
        \centering
        \includegraphics[width=\linewidth]{imágenes/v.png}
    \end{subfigure}
    
    \caption{Imágenes utilizadas para el entrenamiento.}
    \label{fig:imagenes}
\end{figure}

\begin{thebibliography}{9}

\bibitem{autor1}
John Hertz, Anders Krogh, Richard G. Palmer, \textit{Introduction to the Theory of Neural Computation}.

\end{thebibliography}

\end{document}