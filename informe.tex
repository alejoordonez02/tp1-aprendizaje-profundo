\documentclass[a4paper,12pt]{article}

% Paquetes necesarios
\usepackage[utf8]{inputenc}     % Codificación de caracteres
\usepackage[spanish]{babel}     % Idioma del documento
\usepackage{amsmath, amssymb}   % Paquetes para símbolos y ecuaciones matemáticas
\usepackage{graphicx}           % Para insertar imágenes
\usepackage{geometry}           % Control de márgenes
\usepackage{hyperref}           % Enlaces dentro del documento
\geometry{top=2cm, bottom=2cm, left=2cm, right=2cm}

% Datos del documento
\title{Informe de Trabajo Matemático}
\author{Nombre del Autor}
\date{\today}

\begin{document}

\maketitle

\tableofcontents % Tabla de contenido

\newpage

\section{Introducción}

Este informe trata sobre \textit{tema del trabajo}. En esta sección se incluye una breve descripción del problema que se abordará, los objetivos del informe, y el enfoque general que se utilizará.

\section{Desarrollo Matemático}

\subsection{Definiciones y Teoremas}

Aquí puedes comenzar con algunas definiciones formales o teoremas que serán utilizados a lo largo del informe.

\textbf{Definición 1.} \textit{(Nombre de la definición)}: Definición formal del término o concepto.

\subsection{Demostración del Teorema Principal}

Puedes presentar los teoremas principales seguidos de sus demostraciones. Por ejemplo:

\textbf{Teorema 1.} Sea \( f(x) \) una función continua en el intervalo \([a,b]\). Entonces:

\[
\int_a^b f(x) \, dx = F(b) - F(a)
\]

donde \( F(x) \) es una primitiva de \( f(x) \).

\textbf{Demostración.} La demostración sigue de la aplicación del teorema fundamental del cálculo.

\subsection{Ejemplos}

Proporciona ejemplos prácticos que ayuden a ilustrar los conceptos y resultados presentados.

\section{Resultados y Discusión}

En esta sección, se presentan los resultados obtenidos y se discuten sus implicaciones. También puedes añadir gráficos o tablas si es necesario.

\begin{figure}[h]
    \centering
    \includegraphics[width=0.5\textwidth]{nombre_de_la_imagen.png}
    \caption{Descripción de la imagen}
    \label{fig:imagen}
\end{figure}

\section{Conclusión}

En esta sección se resumen las ideas principales y se mencionan posibles aplicaciones o trabajos futuros relacionados con este informe.

\section{Bibliografía}

Utiliza \texttt{bibtex} o inserta referencias de manera manual. A continuación un ejemplo simple:

\begin{thebibliography}{9}

\bibitem{autor1}
Autor1, A. \textit{Título del libro o artículo}. Editorial, Año.

\bibitem{autor2}
Autor2, B. \textit{Título del artículo}. \textit{Nombre de la revista}, Volumen (Año), Páginas.

\end{thebibliography}

\end{document}